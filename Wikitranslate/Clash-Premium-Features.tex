\subsection{premium版本的功能特性}
premium核心是专有的,在内部ci管线上构建

\paragraph{代理}
由于对gvisor的依赖,目前只有premium版本有Wireguard\\

\begin{lstlisting}[breaklines=true,language=yaml,style=yaml]
  proxies:
    - name: "wg"
        type: wireguard
        server: 127.0.0.1
        port: 443
        ip: 172.16.0.2
        # ipv6: your_ipv6
        private-key: eCtXsJZ27+4PbhDkHnB923tkUn2Gj59wZw5wFA75MnU=
        public-key: Cr8hWlKvtDt7nrvf+f0brNQQzabAqrjfBvas9pmowjo=
        # preshared-key: base64
        # remote-dns-resolve: true # 使用`dns` 字段对DNS进行远程解析, 
        默认是 true
        # dns: [1.1.1.1, 8.8.8.8]
        # mtu: 1420
        udp: true
\end{lstlisting}


\paragraph{TUN驱动}


不同于硬件物理网卡,TUN 是完全由软件实现的虚拟网络设备,在功能上 TUN 和物理网卡没有区别,它们同样都是网络设备,都可以设置 IP 地址,而且都属于网络设备管理模块,由网络设备管理模块统一来管理。(说人话就是虚拟网卡)

开启TUN后,clash就能以虚拟网卡的方式就能接管你设备的所有网络流量。

需要开启TUN的话只需要将以下内容添加到配置文件中\\ 


\begin{lstlisting}[breaklines=true,language=yaml,style=yaml]
tun:
  enable: true
  stack: system # 或者 gvisor
  # dns-hijack:
  #   - 8.8.8.8:53
  #   - tcp://8.8.8.8:53
  #   - any:53
  #   - tcp://any:53
  auto-route: true #自动设置全局路由
  auto-detect-interface: true # 与interface-name冲突
\end{lstlisting}

或者
\begin{lstlisting}[breaklines=true,language=yaml,style=yaml]
interface-name: en0

tun:
  enable: true
  stack: system # or gvisor
  # dns-hijack:
  #   - 8.8.8.8:53
  #   - tcp://8.8.8.8:53
  auto-route: true # auto set global route
\end{lstlisting}



对于DNS服务器推荐使用\verb|fake-ip|模式


clash需要更高的权限创建TUN设备


\verb|$sudo ./clash|


然后手动创建默认路由和 DNS 服务器。如果你的设备已经有 TUN 设备,那么 Clash TUN 可能无法正常工作。在这种情况下,可以使用 \verb|fake-ip-filter| 解决问题。

\subparagraph{TIPS:} {\em tun驱动也能在安卓设备上使用,但是它的控制设备是 \verb|/dev/tun| 而非 \verb|/dev/net/tun|,所以你需要创建一个软连接解决这个问题,例如:

\verb|ln -sf /dev/tun /dev/net/tun|}
\subparagraph{NOTE:}{\em \verb|auto-route|和\verb|auto-detect-interface| 仅在macOS、Windows,Linux和Android下可用,接收IPv4流量}

\paragraph{windows下的TUN设置}到\url{https://www.wintun.net}下载最后的发行版,拷贝正确的\verb|wintun.dll|到clash的目录下\\

\begin{lstlisting}[breaklines=true,language=yaml,style=yaml]
tun:
  enable: true
  stack: gvisor # 或者 system
  dns-hijack:
    - 198.18.0.2:53 # 当 `fake-ip-range` 是 198.18.0.1/16, 应劫持
198.18.0.2:53
  auto-route: true # 为Windows自动设置全局路由
  # 建议使用`interface-name`
  auto-detect-interface: true # 自动检测接口, 与 `interface-name` 
冲突
\end{lstlisting}



最后打开clash即可。
\paragraph{脚本} 脚本能够让用户以编程的方式灵活的选择网络流量包的代理方式。


例如:\\

\begin{lstlisting}[breaklines=true,language=yaml,style=yaml]
mode: Script

# 参考https://lancellc.gitbook.io/clash/clash-config-file/script
script:
  code: |
    def main(ctx, metadata):
      ip = metadata["dst_ip"] = ctx.resolve_ip(metadata["host"])
      if ip == "":
        return "DIRECT"

      code = ctx.geoip(ip)
      if code == "LAN" or code == "CN":
        return "DIRECT"

      return "Proxy" # 对于没有被其他脚本匹配的请求的默认策略。
\end{lstlisting}

\subparagraph{NOTE:}{\bfseries 如果你想使用IP规则(IP-CIDR GEOIP),你需要先手动解析IP并且将它分配到元数据里。}

上下文和元数据:\\

\begin{lstlisting}[breaklines=true,language=yaml,style=yaml]
interface Metadata {
  type: string // socks5、http
  network: string // tcp
  host: string
  src_ip: string
  src_port: string
  dst_ip: string
  dst_port: string
}

interface Context {
  resolve_ip: (host: string) => string // ip string
  resolve_process_name: (metadata: Metadata) => string
  resolve_process_path: (metadata: Metadata) => string
  geoip: (ip: string) => string // country code
  log: (log: string) => void
  proxy_providers: Record<string, Array<{ name: string, alive: boolean, 
  delay: number }>>
  rule_providers: Record<string, { match: (metadata: Metadata) => boole
  an }>
}
\end{lstlisting}


\paragraph{脚本快捷方式}
在\verb|rules|里使用脚本
\subparagraph{NOTE:}\verb|src_port|和\verb|dst_port|都被定义为变量\\

\begin{lstlisting}[breaklines=true,language=yaml,style=yaml]
script:
  shortcuts:
    quic: network == 'udp' and dst_port == 443

rules:
  - SCRIPT,quic,REJECT
\end{lstlisting}

\paragraph{功能}这部分介绍它的功能

\begin{lstlisting}[breaklines=true,language=yaml,style=yaml]
type resolve_ip = (host: string) => string // ip 字符串
type in_cidr = (ip: string, cidr: string) => boolean // cidra中的ip
type geoip = (ip: string) => string // 国家代号
type match_provider = (name: string) => boolean // 在规则配置文件里
type resolve_process_name = () => string // 查找进程名 (curl .e.g)
type resolve_process_path = () => string // 查找进程路径 (/usr/bin/curl 
.e.g)
\end{lstlisting}

\paragraph{规则配置文件} 对于代理程序来说,规则配置文件都是等价的,它允许用户加载其他文件的规则并且使整体配置文件更加简洁。这也是一个仅限于Premium的功能

定义一个规则配置文件需要添加\verb|rule_providers|字段到主配置文件:\\

\begin{lstlisting}[breaklines=true,language=yaml,style=yaml]
rule-providers:
  apple:
    behavior: "domain"
    type: http
    url: "url"
    interval: 3600
    path: ./apple.yaml
  microsoft:
    behavior: "domain"
    type: file
    path: /microsoft.yaml
\end{lstlisting}


有三种可用的\verb|behavior|:
\subparagraph{domain}域名\\ \\

\begin{lstlisting}[breaklines=true,language=yaml,style=yaml]
payload:
  - '.blogger.com'
  - '*.*.microsoft.com'
  - 'books.itunes.apple.com'
\end{lstlisting}


\subparagraph{ipcider}ip地址\\

\begin{lstlisting}[breaklines=true,language=yaml,style=yaml]
payload:
  - '192.168.1.0/24'
  - '10.0.0.0.1/32'
\end{lstlisting}


\subparagraph{classical}传统的\\

\begin{lstlisting}[breaklines=true,language=yaml,style=yaml]
payload:
  - DOMAIN-SUFFIX,google.com
  - DOMAIN-KEYWORD,google
  - DOMAIN,ad.com
  - SRC-IP-CIDR,192.168.1.201/32
  - IP-CIDR,127.0.0.0/8
  - GEOIP,CN
  - DST-PORT,80
  - SRC-PORT,7777
  # 这里不需要MATCH
\end{lstlisting}



\begin{lstlisting}[breaklines=true,language=yaml,style=yaml]
# 仅限于Premium
rule-providers:
  apple:
    behavior: "domain" # domain, ipcidr 或者 classical (仅限于premium核心 )
    type: http
    url: "url"
    interval: 3600
    path: ./apple.yaml
  microsoft:
    behavior: "domain"
    type: file
    path: /microsoft.yaml

rules:
  - RULE-SET,apple,REJECT
  - RULE-SET,microsoft,policy
\end{lstlisting}


\paragraph{追踪}\url{https://github.com/Dreamacro/clash-tracing}\\


\begin{lstlisting}[breaklines=true,language=yaml,style=yaml]
profile:
    tracing: true
\end{lstlisting}


\paragraph{eBPF} 这个需要\href{https://github.com/iovisor/bcc/blob/master/INSTALL.md#kernel-configuration}{内核支持},只能捕获NIC的输出流量并且和\verb|auto-route|有冲突。\\


\begin{lstlisting}[breaklines=true,language=yaml,style=yaml]
ebpf:
  redirect-to-tun:
    - eth0
\end{lstlisting}

\paragraph{自动重定向}
单纯在Go语言上使用Linux内核的nftables的特性。它可以在不进行任何网络配置的情况下使用\verb|redir-port|(TCP)。


建议使用TUN来处理TCP流量。与仅使用TUN相比,它提高了一些低性能设备的网络吞吐量性能。


\begin{lstlisting}[breaklines=true,language=yaml,style=yaml]
interface-name: en0

tun:
  enable: true
  stack: system
  dns-hijack:
    - any:53
  auto-redir: true
  auto-route: true
\end{lstlisting}