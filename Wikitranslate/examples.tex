\subsection{代码示例}
这一章节在于给大家一些特殊的示例用法,给大家展示一下这个软件的强大的特殊用法。
\paragraph{基于规则的wireguard} 此功能只能在支持wireguard并且开启此功能的内核上使用。
\verb|Table|选项能够阻止\verb|wg-quick|覆盖默认路由

\subparagraph{example}"\verb|wg0.conf|"

\begin{lstlisting}[breaklines=true,language=sh]
[Interface]
PrivateKey = ...
Address = 172.16.0.1/32
MTU = ...
Table = 6666
PostUp = ip rule add from 172.16.0.1/32 table 6666

[Peer]
AllowedIPs = 0.0.0.0/0
AllowedIPs = ::/0
PublicKey = ...
Endpoint = ...
\end{lstlisting}

那么在Clash中,你只需要创建一个名为“DIRECT”的具有特输出接口代理组。

\begin{lstlisting}[breaklines=true,language=yaml,style=yaml]
proxy-groups:
  - name: Wireguard
    type: select
    interface-name: wg0
    proxies:
      - DIRECT
rules:
  - DOMAIN,google.com,Wireguard
\end{lstlisting}

\paragraph{与OpenConnect一起使用}OpenConnect支持Cisco AnyConnect SSL VPN, Juniper Network Connect, Palo Alto Networks (PAN) GlobalProtect SSL VPN, Pulse Connect Secure SSL VPN, F5 BIG-IP SSL VPN, FortiGate SSL VPN and Array Networks SSL VPN.

例如,当你的公司里使用Cisco AnyConnect进行内部网络访问。在这里我会展示如何使用clash的路由策略进行更方便的上网。

首先,你需要安装\href{https://github.com/dlenski/vpn-slice#requirements}{vpn-slice},这个工具能够覆盖默认的OpenConnect行为。通俗的来说,这个软件能够阻止VPN软件接管你的默认网络路由。

接下来你需要写一个脚本(在下文中我们将其称作为\verb|tun0.sh|)就像这样子:(windows下请自行脑补)


\begin{lstlisting}[breaklines=true,language=yaml,style=yaml]
#!/bin/bash
ANYCONNECT_HOST="vpn.example.com"
ANYCONNECT_USER="john"
ANYCONNECT_PASSWORD="foobar"
ROUTING_TABLE_ID="6667"
TUN_INTERFACE="tun0"

#如果服务器在中国大陆,则需要添加--no-dtls。因为中国的UDP可能是是不稳定的。
echo "$ANYCONNECT_PASSWORD" | \
  openconnect \
    --non-inter \
    --passwd-on-stdin \
    --protocol=anyconnect \
    --interface $TUN_INTERFACE \
    --script "vpn-slice
if [ \"\$reason\" = 'connect' ]; then
  ip rule add from \$INTERNAL_IP4_ADDRESS table $ROUTING_TABLE_ID
  ip route add default dev \$TUNDEV scope link table $ROUTING_TABLE_ID
elif [ \"\$reason\" = 'disconnect' ]; then
  ip rule del from \$INTERNAL_IP4_ADDRESS table $ROUTING_TABLE_ID
  ip route del default dev \$TUNDEV scope link table $ROUTING_TABLE_ID
fi" \
    --user $ANYCONNECT_USER \
    https://$ANYCONNECT_HOST
\end{lstlisting}


然后你需要将其配置为一个systemd服务。

创建\verb|text/etc/systemd/system/tun0.service|

\begin{lstlisting}
Description=Cisco AnyConnect VPN
After=network-online.target
Conflicts=shutdown.target sleep.target

[Service]
Type=simple
ExecStart=/path/to/tun0.sh
KillSignal=SIGINT
Restart=always
RestartSec=3
StartLimitIntervalSec=0

[Install]
WantedBy=multi-user.target
\end{lstlisting}

然乎启动这个服务
\begin{lstlisting}[breaklines=true,language=sh]
chmod +x /path/to/tun0.sh
systemctl daemon-reload
systemctl enable tun0
systemctl start tun0
\end{lstlisting}

然后你可以查看日志查看它有没有正常运行,简单的看是否创建了\verb|tun0|端口。

和Wireguard相似,,将出口连接到TUN设备只需要添加一个代理组:

\begin{lstlisting}[breaklines=true,language=yaml,style=yaml]
proxy-groups:
  - name: Cisco AnyConnect VPN
    type: select
    interface-name: tun0
    proxies:
      - DIRECT
\end{lstlisting}

……然后就可以准备使用了。添加需要的规则:

\begin{lstlisting}[breaklines=true,language=yaml,style=yaml]
rules:
  - DOMAIN-SUFFIX,internal.company.com,Cisco AnyConnect VPN
\end{lstlisting}

当出现一些问题时,你可以查看调试级别的日志。