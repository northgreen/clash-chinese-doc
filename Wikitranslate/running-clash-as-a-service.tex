\subsection{将clash作为服务运行}

\paragraph{序言}clash被设计成在后台运行,但是目前golang下并没有什么完美的方法实现后台守护。我们可以用第三方工具将clash设置为守护进程。

\paragraph{systemd} 将clash的二进制文件拷贝到\verb|/usr/local/bin|,将配置文件放在\verb|/etc/clash|:
\begin{lstlisting}[breaklines=true,language=sh]
$ cp clash /usr/local/bin
$ cp config.yaml /etc/clash/
$ cp Country.mmdb /etc/clash/
\end{lstlisting}

在\verb|/etc/systemd/system/clash.service|创建配置文件

\begin{lstlisting}[breaklines=true]
[Unit]
Description=Clash daemon, A rule-based proxy in Go.
After=network.target

[Service]
Type=simple
Restart=always
ExecStart=/usr/local/bin/clash -d /etc/clash

[Install]
WantedBy=multi-user.target
\end{lstlisting}

然后你需要重新加载systemd:
\begin{lstlisting}[breaklines=true,language=sh]
$ systemctl daemon-reload
\end{lstlisting}
让clashd随系统启动而启动
\begin{lstlisting}[breaklines=true,language=sh]
$ systemctl enable clash
\end{lstlisting}
使用以下命令检查clash的运行情况和日志
\begin{lstlisting}[breaklines=true,language=sh]
$ systemctl status clash
$ journalctl -xe
\end{lstlisting}
(本指南的提供者:\href{https://github.com/ktechmidas}{ktechmidas}(\href{https://github.com/Dreamacro/clash/issues/754}{\#754})

\paragraph{Docker} 如果你使用的是Linux系统的话我们推荐使用\href{https://docs.docker.com/compose/}{docker compose}

在MacOS或者Windows上更推荐使用第三方clash GUI(\href{https://install.appcenter.ms/users/clashx/apps/clashx-pro/distribution_groups/public}{ClashX Pro}或者\href{https://github.com/Fndroid/clash_for_windows_pkg}{clash for Windows})

另外并不推荐在Dockers容器中运行Clash Premium(\href{https://github.com/Dreamacro/clash/issues/2249#issuecomment-1203494599}{\#2249})
\begin{lstlisting}[breaklines=true,language=yaml,style=yaml]
services:
  clash:
    # ghcr.io/dreamacro/clash
    # ghcr.io/dreamacro/clash-premium
    # dreamacro/clash
    # dreamacro/clash-premium
    image: dreamacro/clash
    container_name: clash
    volumes:
      - ./config.yaml:/root/.config/clash/config.yaml
      # - ./ui:/ui # dashboard volume
    ports:
      - "7890:7890"
      - "7891:7891"
      # - "8080:8080" # 外部控制器
    # TUN
    # cap_add:
    #   - NET_ADMIN
    # devices:
    #   - /dev/net/tun
    restart: unless-stopped
    network_mode: "bridge" # 或者是linux上的host
\end{lstlisting}

保存为\verb|docker-compose.yaml|然后在同一目录下创建\verb|config.yaml|,并且运行以下命令启动clash
\begin{lstlisting}[breaklines=true,language=sh]
$ docker-compose up -d
\end{lstlisting}
使用以下命令查看日志
\begin{lstlisting}[breaklines=true,language=sh]
$ docker-compose logs
\end{lstlisting}
使用以下命令停止clash
\begin{lstlisting}[breaklines=true,language=sh]
$ docker-compose stop
\end{lstlisting}

\paragraph{PM2} PM2

\begin{lstlisting}[breaklines=true,language=sh]
$ wget -qO- https://getpm2.com/install.sh | bash
$ pm2 start clash
\end{lstlisting}