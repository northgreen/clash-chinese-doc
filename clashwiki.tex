\documentclass{ctexart}

\usepackage{hyperref}
\usepackage{listings}
\usepackage{xcolor}

% 用来设置代码的样式

\lstset{
    basicstyle          =  \sffamily,          % 基本代码风格
    keywordstyle        =   \color{orange}\bfseries,          % 关键字风格
    commentstyle        =   \rmfamily\itshape,  % 注释的风格,斜体
    stringstyle         =   \color{green}\ttfamily,  % 字符串风格
    flexiblecolumns,                % 别问为什么,加上这个
    numbers             =   left,   % 行号的位置在左边
    showspaces          =   false,  % 是否显示空格,显示了有点乱,所以不现实了
    numberstyle         =   \zihao{-5}\ttfamily,    % 行号的样式,小五号,tt等宽字体
    showstringspaces    =   false,
    captionpos          =   t,      % 这段代码的名字所呈现的位置,t指的是top上面
    frame               =   lrtb,  % 显示边框
}


\lstdefinelanguage{yaml}
{morekeywords={−,:},
sensitive=true,
morecomment=[l]{\#},
morecomment=[l][\color{black}]{:},
}

\lstdefinestyle{yaml}{
     basicstyle=\color{blue},
     rulecolor=\color{black},
     comment=[l][\color{black}]{:},
     morecomment=[l][\color{gray}\rmfamily\itshape]{\#},
     morecomment=[l][\color{black}]{-},
     morestring=[s][\color{green}]{'}{'}
 }



\title{一份不完整的clash教程}
\author{楚天杳-ictye}

\begin{document}
%头部
\maketitle

\section*{编版信息}
\subsubsection*{参与制作人列表}

\begin{table}[h]

\begin{tabular}{c|cc}
\hline
\textbf{名称}&\textbf{贡献}&\textbf{时间}\\
\hline
楚天杳&创建和主要编辑者&(空)\\
(空)&审稿与校订&(空)\\
\hline
\end{tabular}

\end{table}
     




\tableofcontents

\section*{作者的话}
其实我自己也找了很久系统讲解clash的文档,但是好像直接介绍clash所有配置的人很少,所以最终还是自己做吧。

这份文档主要是翻译clash的文档,其实并不复杂,通过机器翻译和我本人的一些修正再加一些修改等等高科技,给大家一个很好的配置clash的参考。

文档错误可以直接通过邮箱northgreen2006@qq.com进行反馈,我也希望有人能直接改正我的错误。

\section*{一些引路的链接}

\begin{itemize}

        \item clash github项目页:\url{https://github.com/Dreamacro/clash}
        \item clash for windows(一个图形界面的clash,全平台支持,傻瓜式操作(不仅仅是Windows系统)。默认是英文的,GitHub上有很多汉化补丁包,很好用。):\url{https://github.com/Fndroid/clash_for_windows_pkg}
        \item yacd(一个网页端的clash控制面板,很像clash for Windows,但是它是运行浏览器里的,通过api来控制clash,很适合在服务器上用):\url{https://github.com/haishanh/yacdURL}

\end{itemize}


%维基翻译章节序章
\section{clash维基的(不完全)翻译}
此部分是对clash维基文档的一个翻译,稍加作者本人的润色和整理。
\subsection{clash介绍}
clash是一个广受大家喜爱的代理软件,目前分为两个版本:
\begin{itemize}
    \item clash:在\href{https://github.com/Dreamacro/clash}{/Dreamacro/clash}上发布的开源软件。
        \item clash premium:带有tun和其他一些功能的clash二进制文件(也是免费的)
    \end{itemize}

\subsubsection{安装}
    直接从\url{https://github.com/Dreamacro/clash/releases}获取预先构建好的二进制文件,然后将其配置到系统的patch中即可(Windows下最直接的方法是直接复制到system32),国内用户会有个问题就是\verb|Country.mmdb|无法下载,这个文件网上也有很多人分享它的链接。当然也可以曲线救国,使用clash for Windows的这个文件,clash for Windows默认自带这个文件。另外你也可以从源码构建二进制文件,这需要自行查看他们自己的\href{https://github.com/Dreamacro/clash/wiki}{wiki}

%维基翻译章节

%将clash作为服务运行
\subsection{将clash作为服务运行}

\paragraph{序言}clash被设计成在后台运行,但是目前golang下并没有什么完美的方法实现后台守护。我们可以用第三方工具将clash设置为守护进程。

\paragraph{systemd} 将clash的二进制文件拷贝到\verb|/usr/local/bin|,将配置文件放在\verb|/etc/clash|:
\begin{lstlisting}[breaklines=true,language=sh]
$ cp clash /usr/local/bin
$ cp config.yaml /etc/clash/
$ cp Country.mmdb /etc/clash/
\end{lstlisting}

在\verb|/etc/systemd/system/clash.service|创建配置文件

\begin{lstlisting}[breaklines=true]
[Unit]
Description=Clash daemon, A rule-based proxy in Go.
After=network.target

[Service]
Type=simple
Restart=always
ExecStart=/usr/local/bin/clash -d /etc/clash

[Install]
WantedBy=multi-user.target
\end{lstlisting}

然后你需要重新加载systemd:
\begin{lstlisting}[breaklines=true,language=sh]
$ systemctl daemon-reload
\end{lstlisting}
让clashd随系统启动而启动
\begin{lstlisting}[breaklines=true,language=sh]
$ systemctl enable clash
\end{lstlisting}
使用以下命令检查clash的运行情况和日志
\begin{lstlisting}[breaklines=true,language=sh]
$ systemctl status clash
$ journalctl -xe
\end{lstlisting}
(本指南的提供者:\href{https://github.com/ktechmidas}{ktechmidas}(\href{https://github.com/Dreamacro/clash/issues/754}{\#754})

\paragraph{Docker} 如果你使用的是Linux系统的话我们推荐使用\href{https://docs.docker.com/compose/}{docker compose}

在MacOS或者Windows上更推荐使用第三方clash GUI(\href{https://install.appcenter.ms/users/clashx/apps/clashx-pro/distribution_groups/public}{ClashX Pro}或者\href{https://github.com/Fndroid/clash_for_windows_pkg}{clash for Windows})

另外并不推荐在Dockers容器中运行Clash Premium(\href{https://github.com/Dreamacro/clash/issues/2249#issuecomment-1203494599}{\#2249})
\begin{lstlisting}[breaklines=true,language=yaml,style=yaml]
services:
  clash:
    # ghcr.io/dreamacro/clash
    # ghcr.io/dreamacro/clash-premium
    # dreamacro/clash
    # dreamacro/clash-premium
    image: dreamacro/clash
    container_name: clash
    volumes:
      - ./config.yaml:/root/.config/clash/config.yaml
      # - ./ui:/ui # dashboard volume
    ports:
      - "7890:7890"
      - "7891:7891"
      # - "8080:8080" # 外部控制器
    # TUN
    # cap_add:
    #   - NET_ADMIN
    # devices:
    #   - /dev/net/tun
    restart: unless-stopped
    network_mode: "bridge" # 或者是linux上的host
\end{lstlisting}

保存为\verb|docker-compose.yaml|然后在同一目录下创建\verb|config.yaml|,并且运行以下命令启动clash
\begin{lstlisting}[breaklines=true,language=sh]
$ docker-compose up -d
\end{lstlisting}
使用以下命令查看日志
\begin{lstlisting}[breaklines=true,language=sh]
$ docker-compose logs
\end{lstlisting}
使用以下命令停止clash
\begin{lstlisting}[breaklines=true,language=sh]
$ docker-compose stop
\end{lstlisting}

\paragraph{PM2} PM2

\begin{lstlisting}[breaklines=true,language=sh]
$ wget -qO- https://getpm2.com/install.sh | bash
$ pm2 start clash
\end{lstlisting}

% 配置文档小章节
\subsection{clash配置文件}

\paragraph{简介}clash使用\href{https://yaml.org/}{YAML}语言,(YAML Ain't Markup Language)写配置文件,YAML语言旨在上易于计算机的解析和人的阅读。在这个章节里我们将会介绍clash的常见功能并且如何使用和配置它们。

clash通过在本地创建一个HTTP、SOCKS5或者透明代理服务器,当一个请求(或者说是数据包)被收到时,clash将会将数据路由到不同的远程代理服务器,这些节点使用VMess、Shadowsocks、Snell、Trojan、SOCKS5或HTTP协议。

\paragraph{所有的配置选项}这里将会列出所有的配置选项。你只需要照着做就行了。\\

\begin{lstlisting}[breaklines=true,language=yaml,style=yaml]
# 本地HTTP(S)代理服务器的端口
port: 7890

# 本地SOCKS5代理服务器的端口
socks-port: 7891

# 在Linux和macOS上的透明代理服务器的端口(重定向TCP和TProxy UDP)
# redir-port: 7892

# Linux下的透明服务器端口(TProxy TCP和TProxy UDP)
# tproxy-port: 7893

# 在同一端口上使用HTTP(S)和SOCKS4(A) 
# mixed-port: 7890

# 本地SOCKS5/HTTP(S)代理服务器的验证密钥
# authentication:
#  - "user1:pass1"
#  - "user2:pass2"

# 设置为true,以允许来自其他局域网IP地址的连接到本地端服务器。
# allow-lan: false

# 只在“allow-lan”为“true”的情况下可用
# 这个选项控制那些局域网ip可以链接到本机。
# '*': 允许所有IP地址
# 192.168.122.11: 允许一个IPv4地址
# "[aaaa::a8aa:ff:fe09:57d8]": 允许一个IPv6地址
# bind-address: '*'

# clash路由策略
# rule: 基于规则的数据包路由模式
# global: 所有的数据包都将转发到单个规则
# direct: 直接向互联网转发数据包
mode: rule

# 一般情况下,clash会将日志输出到标准输入输出流(STDOUT)
# 可选参数:info / warning / error / debug / silent
# log-level: info

# 当这个选项被设置为false时,解释器将不会使用NAT-IPv6
# ipv6: false

# RESTful网络API监听地址。通过这个功能,你能够控制或者开发一个clash的web控制端
external-controller: 127.0.0.1:9090

# 在配置目录的相对路径或放置一些静态web资源的目录的绝对路径。
#Clash的核心将会在“http://{{external controller}}/ui”
#上为其提供web服务器。
#通过这个,你可以将yacd等可以与clash RESTful API对接的网页程序部署
#于此,这是很方便的。
# external-ui: 你的路径

# RESTful API的密钥(可选) 
# 要通过HTTP头`Authorization: Bearer $ {secret}`进行身份验证 
# 如果RETful api 在0.0.0.0监听,则要求验证
# secret: ""

# 接口名字 (Outbound interface name)
# interface-name: en0

# fumark只有在Linux下是可用的
# routing-mark: 6666

# 静态DNS解析(像/etc/host) 
#
# 支持通配符(e.g. *.clash.dev, *.foo.*.example.com)
# 非通配符域名优先级高于通配符域名
# e.g. foo.example.com > *.example.com > .example.com
# P.S. +.foo.com 和 .foo.com 和 foo.com 是平等的。
# hosts:
  # '*.clash.dev': 127.0.0.1
  # '.dev': 127.0.0.1
  # 'alpha.clash.dev': '::1'

# profile:
  # 将`select`的结果保存到$HOME/.config/clash/.cache
  # 如果你并不需要这个,请将它设为false
  # 当两个不同的配置具有相同名称的组时,所选择的值将被共享
  # store-selected: true

  # 保持fakeip
  # store-fake-ip: false

# DNS服务器设置
# 这个选项是可选的,如果没有,将不会设置DNS服务器。
dns:
  enable: false
  listen: 0.0.0.0:53
  # ipv6: false # 当为false时,对AAAA的请求的响应将为空 

  # DNS服务器使用以下名称服务器进行解析 
  # 只需要指定IP地址
  default-nameserver:
    - 114.114.114.114
    - 8.8.8.8
  # enhanced-mode: fake-ip
  fake-ip-range: 198.18.0.1/16 # CIDR假IP地址池
  # use-hosts: true # 查找地址并且返回IP记录
  
  # 访问这里的IP将不会使用虚假IP
  # 也就是说将会向这些IP的访问都会使用
  # 真实IP地址
  # fake-ip-filter:
  #   - '*.lan'
  #   - localhost.ptlogin2.qq.com
  
  # 支持UDP,TCP,DoT,DoH协议。 你能够指定端口。
  # 所有的DNS请求都不经代理发送到解析服务器,Clash将会采取第一个响应,并不会综合起来。 
  nameserver:
    - 114.114.114.114 # default value
    - 8.8.8.8 # default value
    - tls://dns.rubyfish.cn:853 # TLS的DNS
    - https://1.1.1.1/dns-query # HTTPS的DNS
    - dhcp://en0 # 来自DHCP的DNS 
    # - '8.8.8.8#en0'

  # 有`fallback`选项时,DNS服务器将向此段中的服务器以及`nameservers`中的服务器发送并发请求 
  # 当GEOIP国家地理位置不是`CN`(中国)时,将使用`fallback`服务器中的回应。 
  # fallback:
  #   - tcp://1.1.1.1
  #   - 'tcp://1.1.1.1#en0'

  # 如果使用`nameservers`服务器解析得到的IP地址属于指定的子网范围内,则被视为无效地址,DNS服务器将使用`fallback`服务器返回解析结果。 
  # IP address resolved with servers in `nameserver` is used when`fallback-filter.geoip` is true and when GEOIP of the IP address is `CN`.  当`fallback-filter.geoip`为true且IP地址的GEOIP为`CN`时,将使用在`nameserver`中解析的服务器的IP地址。
  #

  #如果`fallback-filter.geoip`为false,且不匹配`fallback-filter.ipcidr`,则始终使用来自`nameserver`命名服务器的结果。
  # 这是对DNS污染攻击的一种防范措施。
  # fallback-filter:
  #   geoip: true
  #   geoip-code: CN
  #   ipcidr:
  #     - 240.0.0.0/4
  #   domain:
  #     - '+.google.com'
  #     - '+.facebook.com'
  #     - '+.youtube.com'
  
  # 通过特定的DNS服务器解析指定域名。
  # nameserver-policy:
  #   'www.baidu.com': '114.114.114.114'
  #   '+.internal.crop.com': '10.0.0.1'

proxies:
  # Shadowsocks
  # 密码支持(加密方法):
  #   aes-128-gcm aes-192-gcm aes-256-gcm
  #   aes-128-cfb aes-192-cfb aes-256-cfb
  #   aes-128-ctr aes-192-ctr aes-256-ctr
  #   rc4-md5 chacha20-ietf xchacha20
  #   chacha20-ietf-poly1305 xchacha20-ietf-poly1305
  - name: "ss1"
    type: ss
    server: server
    port: 443
    cipher: chacha20-ietf-poly1305
    password: "password"
    # udp: true

  - name: "ss2"
    type: ss
    server: server
    port: 443
    cipher: chacha20-ietf-poly1305
    password: "password"
    plugin: obfs
    plugin-opts:
      mode: tls # or http
      # host: bing.com

  - name: "ss3"
    type: ss
    server: server
    port: 443
    cipher: chacha20-ietf-poly1305
    password: "password"
    plugin: v2ray-plugin
    plugin-opts:
      mode: websocket # 暂时没有QUIC
      # tls: true # wss
      # skip-cert-verify: true
      # host: bing.com
      # path: "/"
      # mux: true
      # headers:
      #   custom: value

  # vmess
  # 密码支持auto/aes-128-gcm/chacha20-poly1305/none
  - name: "vmess"
    type: vmess
    server: server
    port: 443
    uuid: uuid
    alterId: 32
    cipher: auto
    # udp: true
    # tls: true
    # skip-cert-verify: true
    # servername: example.com # 优先于wss主机 
    # network: ws
    # ws-opts:
    #   path: /path
    #   headers:
    #     Host: v2ray.com
    #   max-early-data: 2048
    #   early-data-header-name: Sec-WebSocket-Protocol

  - name: "vmess-h2"
    type: vmess
    server: server
    port: 443
    uuid: uuid
    alterId: 32
    cipher: auto
    network: h2
    tls: true
    h2-opts:
      host:
        - http.example.com
        - http-alt.example.com
      path: /
  
  - name: "vmess-http"
    type: vmess
    server: server
    port: 443
    uuid: uuid
    alterId: 32
    cipher: auto
    # udp: true
    # network: http
    # http-opts:
    #   # method: "GET"
    #   # path:
    #   #   - '/'
    #   #   - '/video'
    #   # headers:
    #   #   Connection:
    #   #     - keep-alive

  - name: vmess-grpc
    server: server
    port: 443
    type: vmess
    uuid: uuid
    alterId: 32
    cipher: auto
    network: grpc
    tls: true
    servername: example.com
    # skip-cert-verify: true
    grpc-opts:
      grpc-service-name: "example"

  # socks5
  - name: "socks"
    type: socks5
    server: server
    port: 443
    # username: username
    # password: password
    # tls: true
    # skip-cert-verify: true
    # udp: true

  # http
  - name: "http"
    type: http
    server: server
    port: 443
    # username: username
    # password: password
    # tls: true # https
    # skip-cert-verify: true
    # sni: custom.com

  # Snell
  # 注意,暂时还不支持UDP协议 
  - name: "snell"
    type: snell
    server: server
    port: 44046
    psk: yourpsk
    # version: 2
    # obfs-opts:
      # mode: http # 或者tls
      # host: bing.com

  # Trojan
  - name: "trojan"
    type: trojan
    server: server
    port: 443
    password: yourpsk
    # udp: true
    # sni: example.com # 也可以成为服务器名称
    # alpn:
    #   - h2
    #   - http/1.1
    # skip-cert-verify: true

  - name: trojan-grpc
    server: server
    port: 443
    type: trojan
    password: "example"
    network: grpc
    sni: example.com
    # skip-cert-verify: true
    udp: true
    grpc-opts:
      grpc-service-name: "example"

  - name: trojan-ws
    server: server
    port: 443
    type: trojan
    password: "example"
    network: ws
    sni: example.com
    # skip-cert-verify: true
    udp: true
    # ws-opts:
      # path: /path
      # headers:
      #   Host: example.com

  # ShadowsocksR
  # 支持的密码(加密方式):ss中所有的流密码 
  # 支持的混淆方式(obfses):
  #   plain http_simple http_post
  #   random_head tls1.2_ticket_auth tls1.2_ticket_fastauth
  # The supported supported protocols:
  #   origin auth_sha1_v4 auth_aes128_md5
  #   auth_aes128_sha1 auth_chain_a auth_chain_b  
  - name: "ssr"
    type: ssr
    server: server
    port: 443
    cipher: chacha20-ietf
    password: "password"
    obfs: tls1.2_ticket_auth
    protocol: auth_sha1_v4
    # obfs-param: domain.tld
    # protocol-param: "#"
    # udp: true

proxy-groups:
  # relay chains the proxies. proxies shall not contain a relay. No UDP support.
  # Traffic: clash <-> http <-> vmess <-> ss1 <-> ss2 <-> Internet
  - name: "relay"
    type: relay
    proxies:
      - http
      - vmess
      - ss1
      - ss2

  # url-test会通过对代理进行URL的标准测试进行测速来选择使用哪个代理
  - name: "auto"
    type: url-test
    proxies:
      - ss1
      - ss2
      - vmess1
    # tolerance: 150
    # lazy: true
    url: 'http://www.gstatic.com/generate_204'
    interval: 300

  # fallback会按照优先级选择一个可用的策略。通过一个URL对代理服务器进行测试,就像url-test组一样。
  - name: "fallback-auto"
    type: fallback
    proxies:
      - ss1
      - ss2
      - vmess1
    url: 'http://www.gstatic.com/generate_204'
    interval: 300

  # 负载均衡:相同的eTLD+1请求将被拨打到同一个代理。 load-balance: The request of the same eTLD+1 will be dial to the same proxy.
  - name: "load-balance"
    type: load-balance
    proxies:
      - ss1
      - ss2
      - vmess1
    url: 'http://www.gstatic.com/generate_204'
    interval: 300
    #  strategy: consistent-hashing # 或者 round-robin

  # select用于选择代理或代理组。建议使用RESTful API切换代理,适合在GUI中的使用。 
  - name: Proxy
    type: select
    # disable-udp: true
    proxies:
      - ss1
      - ss2
      - vmess1
      - auto
 
  # 可以将流量定向到另一个接口名或者fwmark,也支持在代理中使用。
  - name: en1
    type: select
    interface-name: en1
    routing-mark: 6667
    proxies:
      - DIRECT 

  - name: UseProvider
    type: select
    use:
      - provider1
    proxies:
      - Proxy
      - DIRECT

proxy-providers:
  provider1:
    type: http
    url: "url"
    interval: 3600
    path: ./provider1.yaml
    health-check:
      enable: true
      interval: 600
      # lazy: true
      url: http://www.gstatic.com/generate_204
  test:
    type: file
    path: /test.yaml
    health-check:
      enable: true
      interval: 36000
      url: http://www.gstatic.com/generate_204

tunnels:
  # 一行的配置
  - tcp/udp,127.0.0.1:6553,114.114.114.114:53,proxy
  - tcp,127.0.0.1:6666,rds.mysql.com:3306,vpn
  # 完整的yaml配置
  - network: [tcp, udp]
    address: 127.0.0.1:7777
    target: target.com
    proxy: proxy

rules:
  - DOMAIN-SUFFIX,google.com,auto
  - DOMAIN-KEYWORD,google,auto
  - DOMAIN,google.com,auto
  - DOMAIN-SUFFIX,ad.com,REJECT
  - SRC-IP-CIDR,192.168.1.201/32,DIRECT
  # 可选的参数“no-resolve”是针对IP规则的
  - IP-CIDR,127.0.0.0/8,DIRECT
  - GEOIP,CN,DIRECT
  - DST-PORT,80,DIRECT
  - SRC-PORT,7777,DIRECT
  - RULE-SET,apple,REJECT # 仅限于Premium版本
  - MATCH,auto
\end{lstlisting}


%代码示例小章节
\subsection{代码示例}
这一章节在于给大家一些特殊的示例用法,给大家展示一下这个软件的强大的特殊用法。
\paragraph{基于规则的wireguard} 此功能只能在支持wireguard并且开启此功能的内核上使用。
\verb|Table|选项能够阻止\verb|wg-quick|覆盖默认路由

\subparagraph{example}"\verb|wg0.conf|"

\begin{lstlisting}[breaklines=true,language=sh]
[Interface]
PrivateKey = ...
Address = 172.16.0.1/32
MTU = ...
Table = 6666
PostUp = ip rule add from 172.16.0.1/32 table 6666

[Peer]
AllowedIPs = 0.0.0.0/0
AllowedIPs = ::/0
PublicKey = ...
Endpoint = ...
\end{lstlisting}

那么在Clash中,你只需要创建一个名为“DIRECT”的具有特输出接口代理组。

\begin{lstlisting}[breaklines=true,language=yaml,style=yaml]
proxy-groups:
  - name: Wireguard
    type: select
    interface-name: wg0
    proxies:
      - DIRECT
rules:
  - DOMAIN,google.com,Wireguard
\end{lstlisting}

\paragraph{与OpenConnect一起使用}OpenConnect支持Cisco AnyConnect SSL VPN, Juniper Network Connect, Palo Alto Networks (PAN) GlobalProtect SSL VPN, Pulse Connect Secure SSL VPN, F5 BIG-IP SSL VPN, FortiGate SSL VPN and Array Networks SSL VPN.

例如,当你的公司里使用Cisco AnyConnect进行内部网络访问。在这里我会展示如何使用clash的路由策略进行更方便的上网。

首先,你需要安装\href{https://github.com/dlenski/vpn-slice#requirements}{vpn-slice},这个工具能够覆盖默认的OpenConnect行为。通俗的来说,这个软件能够阻止VPN软件接管你的默认网络路由。

接下来你需要写一个脚本(在下文中我们将其称作为\verb|tun0.sh|)就像这样子:(windows下请自行脑补)


\begin{lstlisting}[breaklines=true,language=yaml,style=yaml]
#!/bin/bash
ANYCONNECT_HOST="vpn.example.com"
ANYCONNECT_USER="john"
ANYCONNECT_PASSWORD="foobar"
ROUTING_TABLE_ID="6667"
TUN_INTERFACE="tun0"

#如果服务器在中国大陆,则需要添加--no-dtls。因为中国的UDP可能是是不稳定的。
echo "$ANYCONNECT_PASSWORD" | \
  openconnect \
    --non-inter \
    --passwd-on-stdin \
    --protocol=anyconnect \
    --interface $TUN_INTERFACE \
    --script "vpn-slice
if [ \"\$reason\" = 'connect' ]; then
  ip rule add from \$INTERNAL_IP4_ADDRESS table $ROUTING_TABLE_ID
  ip route add default dev \$TUNDEV scope link table $ROUTING_TABLE_ID
elif [ \"\$reason\" = 'disconnect' ]; then
  ip rule del from \$INTERNAL_IP4_ADDRESS table $ROUTING_TABLE_ID
  ip route del default dev \$TUNDEV scope link table $ROUTING_TABLE_ID
fi" \
    --user $ANYCONNECT_USER \
    https://$ANYCONNECT_HOST
\end{lstlisting}


然后你需要将其配置为一个systemd服务。

创建\verb|text/etc/systemd/system/tun0.service|

\begin{lstlisting}
Description=Cisco AnyConnect VPN
After=network-online.target
Conflicts=shutdown.target sleep.target

[Service]
Type=simple
ExecStart=/path/to/tun0.sh
KillSignal=SIGINT
Restart=always
RestartSec=3
StartLimitIntervalSec=0

[Install]
WantedBy=multi-user.target
\end{lstlisting}

然乎启动这个服务
\begin{lstlisting}[breaklines=true,language=sh]
chmod +x /path/to/tun0.sh
systemctl daemon-reload
systemctl enable tun0
systemctl start tun0
\end{lstlisting}

然后你可以查看日志查看它有没有正常运行,简单的看是否创建了\verb|tun0|端口。

和Wireguard相似,,将出口连接到TUN设备只需要添加一个代理组:

\begin{lstlisting}[breaklines=true,language=yaml,style=yaml]
proxy-groups:
  - name: Cisco AnyConnect VPN
    type: select
    interface-name: tun0
    proxies:
      - DIRECT
\end{lstlisting}

……然后就可以准备使用了。添加需要的规则:

\begin{lstlisting}[breaklines=true,language=yaml,style=yaml]
rules:
  - DOMAIN-SUFFIX,internal.company.com,Cisco AnyConnect VPN
\end{lstlisting}

当出现一些问题时,你可以查看调试级别的日志。

%外接控制器API参考
%External-controller-API-Reference
\subsection{外接控制器api}

\paragraph{介绍} 外部控制器能够让用户程序化地控制clash通过HTTP RESTful API。很大程度上,第三方的clash图形界面都是基于这个功能。通过\verb|external-controller|指定一个地址开启这个功能。

\subparagraph{TIP:}此文档不算太完善,请看\url{http://clash.gitbook.io/doc/restful-api}。

\paragraph{身份验证}
\begin{itemize}
    \item 外部控制器接受\verb|Bearer Tokens|作为验证访问的方式。
        \begin{itemize}
            \item 使用\verb|Bearer <你的密钥>|作为请求的标头,以便传递网络凭据。
        \end{itemize}
\end{itemize}

\paragraph{RESTful API 文档}

\subparagraph{日志(Logs)} 
\begin{itemize}
    \item \verb|/logs|
        \begin{itemize}
            \item 请求方法:\verb|GET|
                \begin{itemize}
                    \item 完整路径:\verb|GET /logs|
                    \item 说明:得到实时的日志
                \end{itemize}
        \end{itemize}
\end{itemize}

\subparagraph{流量(Traffic)}
\begin{itemize}
    \item \verb|/traffic|
    \begin{itemize}
        \item 请求方法:GET
        \begin{itemize}
            \item 完整路径:\verb|GET /traffic|
            \item 描述:获取实时的流量数据
        \end{itemize}
    \end{itemize}
\end{itemize}

\subparagraph{版本(Version)}
\begin{itemize}
    \item \verb|/version|
    \begin{itemize}
        \item 请求方法:\verb|GET|
        \begin{itemize}
            \item 完整路径:\verb|GET /version|
            \item 描述:获得clash版本
        \end{itemize}
    \end{itemize}
\end{itemize}

\subparagraph{配置(Configs)}
\begin{itemize}
    \item \verb|/configs|
    \begin{itemize}
        \item 请求方法:\verb|GET|
        \begin{itemize}
            \item 完整路径:\verb|GET /configs|
            \item 描述:获取基础的配置信息
        \end{itemize}
        \item 请求方法:\verb|PUT|
        \begin{itemize}
            \item 完整路径:\verb|PUT /configs|
            \item 描述:重新加载基础配置
        \end{itemize}
        \item 请求方法:\verb|PATCH|
        \begin{itemize}
            \item 完整路径:\verb|PATCH /configs|
            \item 描述:更新基础配置
        \end{itemize}
    \end{itemize}
\end{itemize}

\subparagraph{代理(proxies)}
\begin{itemize}
    \item \verb|/proxies|
    \begin{itemize}
        \item 请求方法:\verb|GET|
        \begin{itemize}
            \item 完整路径:\verb|GET /proxies|
            \item 描述:获得特定的代理信息
        \end{itemize}

        \item 请求方法:\verb|PUT|
        \begin{itemize}
            \item 完整路径:\verb|PUT /proxies/:name|
            \item 描述:选择具体的代理
        \end{itemize}
    \end{itemize}
    \item \verb|/proxies/:name/delay|
    \begin{itemize}
        \item 请求方法:\verb|GET|
        \begin{itemize}
            \item 完整路径:\verb|GET /proxies/:name/delay|
            \item 描述:获取特定的代理的延迟测试信息
        \end{itemize}
    \end{itemize}
\end{itemize}

\subparagraph{规则(Rules)}
\begin{itemize}
    \item \verb|/rules|
    \begin{itemize}
        \item 请求方法:\verb|GET|
        \begin{itemize}
            \item 完整路径:\verb|GET /rules|
            \item 得到规则的详细信息
        \end{itemize}
    \end{itemize}
\end{itemize}

\subparagraph{连接(connections)}

\begin{itemize}
    \item \verb|/connections|
    \begin{itemize}
        \item 请求方法:\verb|GET|
        \begin{itemize}
            \item 完整路径:\verb|GET /connections|
            \item 描述:获取连接信息
        \end{itemize}
        \item 请求方法:\verb|DELETE|
        \begin{itemize}
            \item 完整路径:\verb|DELETE /connections|
            \item 描述:关闭所有链接
        \end{itemize}
    \end{itemize}
    \item \verb|/connections/:id|
    \begin{itemize}
        \item 请求方法:\verb|DELETE|
        \begin{itemize}
            \item 完整路径:\verb|DELETE /connections/:id|
            \item 描述:闭特定连接
        \end{itemize}
    \end{itemize}
\end{itemize}

\subparagraph{供应商(Providers)}

\begin{itemize}
    \item \verb|/providers/proxies|
    \begin{itemize}
        \item 请求方法:\verb|GET|
        \begin{itemize}
            \item 完整路径\verb|GET /providers/proxies|
            \item 描述:获取所有的代理的信息
        \end{itemize}
    \end{itemize}
    \item \verb|/providers/proxies/:name|
    \begin{itemize}
        \item 请求方法:\verb|GET|
        \begin{itemize}
            \item 完整路径\verb|GET /providers/proxies/:name|
            \item 描述:获取特定的代理提供商的代理的信息
        \end{itemize}
        \item \verb|PUT|
        \begin{itemize}
            \item 完整路径:\verb|GET /providers/proxies/:name|
            \item 描述:选择特定的代理提供商
        \end{itemize}
    \end{itemize}
    \item \verb|/providers/proxies/:name/healthcheck|
    \begin{itemize}
        \item 请求方法:\verb|GET|
        \begin{itemize}
            \item 完整路径\verb|GET /providers/proxies/:name/healthcheck|
            \item 描述:获取特定的代理提供商的代理信息
        \end{itemize}
    \end{itemize}
\end{itemize}

\subparagraph{DNS查询(DNS Query)}

\begin{itemize}
    \item \verb|/dns/query|
    \begin{itemize}
        \item 请求方法:\verb|GET|
        \begin{itemize}
            \item 完整路径\verb|GET /providers/proxies|
            \item 描述:获取指定名称和类型的DNS查询数据
            \item 参数:
            \begin{itemize}
                \item \verb|name|(必选):所要查询的域名
                \item \verb|type|(可选):查询DNS的记录类型(例如: A, MX, CNAME等)如果没有指定的话默认是\verb|A|
            \end{itemize}
        \end{itemize}
        \item 示例:\verb|GET /dns/query?name=example.com&type=A|
    \end{itemize}
\end{itemize}


%premium版本的功能特性章节
\subsection{premium版本的功能特性}
premium核心是专有的,在内部ci管线上构建

\paragraph{代理}
由于对gvisor的依赖,目前只有premium版本有Wireguard\\

\begin{lstlisting}[breaklines=true,language=yaml,style=yaml]
  proxies:
    - name: "wg"
        type: wireguard
        server: 127.0.0.1
        port: 443
        ip: 172.16.0.2
        # ipv6: your_ipv6
        private-key: eCtXsJZ27+4PbhDkHnB923tkUn2Gj59wZw5wFA75MnU=
        public-key: Cr8hWlKvtDt7nrvf+f0brNQQzabAqrjfBvas9pmowjo=
        # preshared-key: base64
        # remote-dns-resolve: true # 使用`dns` 字段对DNS进行远程解析, 
        默认是 true
        # dns: [1.1.1.1, 8.8.8.8]
        # mtu: 1420
        udp: true
\end{lstlisting}


\paragraph{TUN驱动}


不同于硬件物理网卡,TUN 是完全由软件实现的虚拟网络设备,在功能上 TUN 和物理网卡没有区别,它们同样都是网络设备,都可以设置 IP 地址,而且都属于网络设备管理模块,由网络设备管理模块统一来管理。(说人话就是虚拟网卡)

开启TUN后,clash就能以虚拟网卡的方式就能接管你设备的所有网络流量。

需要开启TUN的话只需要将以下内容添加到配置文件中\\ 


\begin{lstlisting}[breaklines=true,language=yaml,style=yaml]
tun:
  enable: true
  stack: system # 或者 gvisor
  # dns-hijack:
  #   - 8.8.8.8:53
  #   - tcp://8.8.8.8:53
  #   - any:53
  #   - tcp://any:53
  auto-route: true #自动设置全局路由
  auto-detect-interface: true # 与interface-name冲突
\end{lstlisting}

或者
\begin{lstlisting}[breaklines=true,language=yaml,style=yaml]
interface-name: en0

tun:
  enable: true
  stack: system # or gvisor
  # dns-hijack:
  #   - 8.8.8.8:53
  #   - tcp://8.8.8.8:53
  auto-route: true # auto set global route
\end{lstlisting}



对于DNS服务器推荐使用\verb|fake-ip|模式


clash需要更高的权限创建TUN设备


\verb|$sudo ./clash|


然后手动创建默认路由和 DNS 服务器。如果你的设备已经有 TUN 设备,那么 Clash TUN 可能无法正常工作。在这种情况下,可以使用 \verb|fake-ip-filter| 解决问题。

\subparagraph{TIPS:} {\em tun驱动也能在安卓设备上使用,但是它的控制设备是 \verb|/dev/tun| 而非 \verb|/dev/net/tun|,所以你需要创建一个软连接解决这个问题,例如:

\verb|ln -sf /dev/tun /dev/net/tun|}
\subparagraph{NOTE:}{\em \verb|auto-route|和\verb|auto-detect-interface| 仅在macOS、Windows,Linux和Android下可用,接收IPv4流量}

\paragraph{windows下的TUN设置}到\url{https://www.wintun.net}下载最后的发行版,拷贝正确的\verb|wintun.dll|到clash的目录下\\

\begin{lstlisting}[breaklines=true,language=yaml,style=yaml]
tun:
  enable: true
  stack: gvisor # 或者 system
  dns-hijack:
    - 198.18.0.2:53 # 当 `fake-ip-range` 是 198.18.0.1/16, 应劫持
198.18.0.2:53
  auto-route: true # 为Windows自动设置全局路由
  # 建议使用`interface-name`
  auto-detect-interface: true # 自动检测接口, 与 `interface-name` 
冲突
\end{lstlisting}



最后打开clash即可。
\paragraph{脚本} 脚本能够让用户以编程的方式灵活的选择网络流量包的代理方式。


例如:\\

\begin{lstlisting}[breaklines=true,language=yaml,style=yaml]
mode: Script

# 参考https://lancellc.gitbook.io/clash/clash-config-file/script
script:
  code: |
    def main(ctx, metadata):
      ip = metadata["dst_ip"] = ctx.resolve_ip(metadata["host"])
      if ip == "":
        return "DIRECT"

      code = ctx.geoip(ip)
      if code == "LAN" or code == "CN":
        return "DIRECT"

      return "Proxy" # 对于没有被其他脚本匹配的请求的默认策略。
\end{lstlisting}

\subparagraph{NOTE:}{\bfseries 如果你想使用IP规则(IP-CIDR GEOIP),你需要先手动解析IP并且将它分配到元数据里。}

上下文和元数据:\\

\begin{lstlisting}[breaklines=true,language=yaml,style=yaml]
interface Metadata {
  type: string // socks5、http
  network: string // tcp
  host: string
  src_ip: string
  src_port: string
  dst_ip: string
  dst_port: string
}

interface Context {
  resolve_ip: (host: string) => string // ip string
  resolve_process_name: (metadata: Metadata) => string
  resolve_process_path: (metadata: Metadata) => string
  geoip: (ip: string) => string // country code
  log: (log: string) => void
  proxy_providers: Record<string, Array<{ name: string, alive: boolean, 
  delay: number }>>
  rule_providers: Record<string, { match: (metadata: Metadata) => boole
  an }>
}
\end{lstlisting}


\paragraph{脚本快捷方式}
在\verb|rules|里使用脚本
\subparagraph{NOTE:}\verb|src_port|和\verb|dst_port|都被定义为变量\\

\begin{lstlisting}[breaklines=true,language=yaml,style=yaml]
script:
  shortcuts:
    quic: network == 'udp' and dst_port == 443

rules:
  - SCRIPT,quic,REJECT
\end{lstlisting}

\paragraph{功能}这部分介绍它的功能

\begin{lstlisting}[breaklines=true,language=yaml,style=yaml]
type resolve_ip = (host: string) => string // ip 字符串
type in_cidr = (ip: string, cidr: string) => boolean // cidra中的ip
type geoip = (ip: string) => string // 国家代号
type match_provider = (name: string) => boolean // 在规则配置文件里
type resolve_process_name = () => string // 查找进程名 (curl .e.g)
type resolve_process_path = () => string // 查找进程路径 (/usr/bin/curl 
.e.g)
\end{lstlisting}

\paragraph{规则配置文件} 对于代理程序来说,规则配置文件都是等价的,它允许用户加载其他文件的规则并且使整体配置文件更加简洁。这也是一个仅限于Premium的功能

定义一个规则配置文件需要添加\verb|rule_providers|字段到主配置文件:\\

\begin{lstlisting}[breaklines=true,language=yaml,style=yaml]
rule-providers:
  apple:
    behavior: "domain"
    type: http
    url: "url"
    interval: 3600
    path: ./apple.yaml
  microsoft:
    behavior: "domain"
    type: file
    path: /microsoft.yaml
\end{lstlisting}


有三种可用的\verb|behavior|:
\subparagraph{domain}域名\\ \\

\begin{lstlisting}[breaklines=true,language=yaml,style=yaml]
payload:
  - '.blogger.com'
  - '*.*.microsoft.com'
  - 'books.itunes.apple.com'
\end{lstlisting}


\subparagraph{ipcider}ip地址\\

\begin{lstlisting}[breaklines=true,language=yaml,style=yaml]
payload:
  - '192.168.1.0/24'
  - '10.0.0.0.1/32'
\end{lstlisting}


\subparagraph{classical}传统的\\

\begin{lstlisting}[breaklines=true,language=yaml,style=yaml]
payload:
  - DOMAIN-SUFFIX,google.com
  - DOMAIN-KEYWORD,google
  - DOMAIN,ad.com
  - SRC-IP-CIDR,192.168.1.201/32
  - IP-CIDR,127.0.0.0/8
  - GEOIP,CN
  - DST-PORT,80
  - SRC-PORT,7777
  # 这里不需要MATCH
\end{lstlisting}



\begin{lstlisting}[breaklines=true,language=yaml,style=yaml]
# 仅限于Premium
rule-providers:
  apple:
    behavior: "domain" # domain, ipcidr 或者 classical (仅限于premium核心 )
    type: http
    url: "url"
    interval: 3600
    path: ./apple.yaml
  microsoft:
    behavior: "domain"
    type: file
    path: /microsoft.yaml

rules:
  - RULE-SET,apple,REJECT
  - RULE-SET,microsoft,policy
\end{lstlisting}


\paragraph{追踪}\url{https://github.com/Dreamacro/clash-tracing}\\


\begin{lstlisting}[breaklines=true,language=yaml,style=yaml]
profile:
    tracing: true
\end{lstlisting}


\paragraph{eBPF} 这个需要\href{https://github.com/iovisor/bcc/blob/master/INSTALL.md#kernel-configuration}{内核支持},只能捕获NIC的输出流量并且和\verb|auto-route|有冲突。\\


\begin{lstlisting}[breaklines=true,language=yaml,style=yaml]
ebpf:
  redirect-to-tun:
    - eth0
\end{lstlisting}

\paragraph{自动重定向}
单纯在Go语言上使用Linux内核的nftables的特性。它可以在不进行任何网络配置的情况下使用\verb|redir-port|(TCP)。


建议使用TUN来处理TCP流量。与仅使用TUN相比,它提高了一些低性能设备的网络吞吐量性能。


\begin{lstlisting}[breaklines=true,language=yaml,style=yaml]
interface-name: en0

tun:
  enable: true
  stack: system
  dns-hijack:
    - any:53
  auto-redir: true
  auto-route: true
\end{lstlisting}

%faq文档
\subsection{FAQ}

\paragraph{error:unsupported rule type RULE-SET}(不支持的规则类型RULE-SET)
\begin{lstlisting}[breaklines=true,language=sh]
FATA[0000] Parse config error: Rules[0] [RULE-SET,apple,REJECT] error: unsupported rule type RULE-SET
\end{lstlisting}
你是用的是clash的开源版本,但是Rule Providers(规则提供程序)目前只能在Premium(免费的)内核版本上使用

\paragraph{我想要VLESS支持}概述:不可能,绝对不可能(bushi)论述:\url{https://github.com/Dreamacro/clash/issues/1185}



\end{document}