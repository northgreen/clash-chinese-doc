% 用来设置代码的样式

\lstset{
    basicstyle          =  \sffamily,          % 基本代码风格
    keywordstyle        =   \color{orange}\bfseries,          % 关键字风格
    commentstyle        =   \rmfamily\itshape,  % 注释的风格,斜体
    stringstyle         =   \color{green}\ttfamily,  % 字符串风格
    flexiblecolumns,                % 别问为什么,加上这个
    numbers             =   left,   % 行号的位置在左边
    showspaces          =   false,  % 是否显示空格,显示了有点乱,所以不现实了
    numberstyle         =   \zihao{-5}\ttfamily,    % 行号的样式,小五号,tt等宽字体
    showstringspaces    =   false,
    captionpos          =   t,      % 这段代码的名字所呈现的位置,t指的是top上面
    frame               =   lrtb,  % 显示边框
}


\lstdefinelanguage{yaml}
{morekeywords={−,:},
sensitive=true,
morecomment=[l]{\#},
morecomment=[l][\color{black}]{:},
}

\lstdefinestyle{yaml}{
     basicstyle=\color{blue},
     rulecolor=\color{black},
     comment=[l][\color{black}]{:},
     morecomment=[l][\color{gray}\rmfamily\itshape]{\#},
     morecomment=[l][\color{black}]{-},
     morestring=[s][\color{green}]{'}{'}
 }

