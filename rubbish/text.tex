\documentclass{ctexart}
\usepackage{hyperref}
\usepackage{tcolorbox}
\tcbuselibrary{breakable}
\usepackage{listings}
\usepackage{xcolor}
% 用来设置附录中代码的样式

\lstset{
    basicstyle          =  \sffamily,          % 基本代码风格
    keywordstyle        =   \color{orange}\bfseries,          % 关键字风格
    commentstyle        =   \rmfamily\itshape,  % 注释的风格,斜体
    stringstyle         =   \color{green}\ttfamily,  % 字符串风格
    flexiblecolumns,                % 别问为什么,加上这个
    numbers             =   left,   % 行号的位置在左边
    showspaces          =   false,  % 是否显示空格,显示了有点乱,所以不现实了
    numberstyle         =   \zihao{-5}\ttfamily,    % 行号的样式,小五号,tt等宽字体
    showstringspaces    =   false,
    captionpos          =   t,      % 这段代码的名字所呈现的位置,t指的是top上面
    frame               =   lrtb,  % 显示边框
}


\lstdefinelanguage{yaml}
{morekeywords={−,:,-},
sensitive=true,
morecomment=[l]{\#},
morecomment=[l][\color{black}]{:},
}

\lstdefinestyle{yaml}{
     basicstyle=\color{blue},
     rulecolor=\color{black},
     comment=[l][\color{black}]{:},
     morecomment=[l][\color{gray}\rmfamily\itshape]{\#},
     morecomment=[l][\color{black}]{-},
     morestring=[s][\color{green}]{'}{'}
 }



\title{一份不完整的clash教程}
\author{楚天杳-ictye}

\begin{document}



\begin{lstlisting}[breaklines=true,language=yaml,style=yaml]
# 本地HTTP(S)代理服务器的端口
port: 7890

# 本地SOCKS5代理服务器的端口
socks-port: 7891

# 在Linux和macOS上的透明代理服务器的端口(重定向TCP和TProxy UDP)
# redir-port: 7892

# Linux下的透明服务器端口(TProxy TCP和TProxy UDP)
# tproxy-port: 7893

# 在同一端口上使用HTTP(S)和SOCKS4(A) 
# mixed-port: 7890

# 本地SOCKS5/HTTP(S)代理服务器的验证密钥
# authentication:
#  - "user1:pass1"
#  - "user2:pass2"

# 设置为true,以允许来自其他局域网IP地址的连接到本地端服务器。
# allow-lan: false

# 只在“allow-lan”为“true”的情况下可用
# 这个选项控制那些局域网ip可以链接到本机。
# '*': 允许所有IP地址
# 192.168.122.11: 允许一个IPv4地址
# "[aaaa::a8aa:ff:fe09:57d8]": 允许一个IPv6地址
# bind-address: '*'

# clash路由策略
# rule: 基于规则的数据包路由模式
# global: 所有的数据包都将转发到单个规则
# direct: 直接向互联网转发数据包
mode: rule

# 一般情况下,clash会将日志输出到标准输入输出流(STDOUT)
# 可选参数:info / warning / error / debug / silent
# log-level: info

# 当这个选项被设置为false时,解释器将不会使用NAT-IPv6
# ipv6: false

# RESTful网络API监听地址。通过这个功能,你能够控制或者开发一个clash的web控制端
external-controller: 127.0.0.1:9090

# 在配置目录的相对路径或放置一些静态web资源的目录的绝对路径。
#Clash的核心将会在“http://{{external controller}}/ui”
#上为其提供web服务器。
#通过这个,你可以将yacd等可以与clash RESTful API对接的网页程序部署
#于此,这是很方便的。
# external-ui: 你的路径

# RESTful API的密钥(可选) 
# 要通过HTTP头`Authorization: Bearer $ {secret}`进行身份验证 Authenticate by spedifying HTTP header `Authorization: Bearer ${secret}`
# ALWAYS set a secret if RESTful API is listening on 0.0.0.0
# secret: ""

# Outbound interface name
# interface-name: en0

# fwmark on Linux only
# routing-mark: 6666

# Static hosts for DNS server and connection establishment (like /etc/hosts)
#
# Wildcard hostnames are supported (e.g. *.clash.dev, *.foo.*.example.com)
# Non-wildcard domain names have a higher priority than wildcard domain names
# e.g. foo.example.com > *.example.com > .example.com
# P.S. +.foo.com equals to .foo.com and foo.com
# hosts:
  # '*.clash.dev': 127.0.0.1
  # '.dev': 127.0.0.1
  # 'alpha.clash.dev': '::1'

# profile:
  # Store the `select` results in $HOME/.config/clash/.cache
  # set false If you don't want this behavior
  # when two different configurations have groups with the same name, the selected values are shared
  # store-selected: true

  # persistence fakeip
  # store-fake-ip: false

# DNS server settings
# This section is optional. When not present, the DNS server will be disabled.
dns:
  enable: false
  listen: 0.0.0.0:53
  # ipv6: false # when the false, response to AAAA questions will be empty

  # These nameservers are used to resolve the DNS nameserver hostnames below.
  # Specify IP addresses only
  default-nameserver:
    - 114.114.114.114
    - 8.8.8.8
  # enhanced-mode: fake-ip
  fake-ip-range: 198.18.0.1/16 # Fake IP addresses pool CIDR
  # use-hosts: true # lookup hosts and return IP record
  
  # Hostnames in this list will not be resolved with fake IPs
  # i.e. questions to these domain names will always be answered with their
  # real IP addresses
  # fake-ip-filter:
  #   - '*.lan'
  #   - localhost.ptlogin2.qq.com
  
  # Supports UDP, TCP, DoT, DoH. You can specify the port to connect to.
  # All DNS questions are sent directly to the nameserver, without proxies
  # involved. Clash answers the DNS question with the first result gathered.
  nameserver:
    - 114.114.114.114 # default value
    - 8.8.8.8 # default value
    - tls://dns.rubyfish.cn:853 # DNS over TLS
    - https://1.1.1.1/dns-query # DNS over HTTPS
    - dhcp://en0 # dns from dhcp
    # - '8.8.8.8#en0'

  # When `fallback` is present, the DNS server will send concurrent requests
  # to the servers in this section along with servers in `nameservers`.
  # The answers from fallback servers are used when the GEOIP country
  # is not `CN`.
  # fallback:
  #   - tcp://1.1.1.1
  #   - 'tcp://1.1.1.1#en0'

  # If IP addresses resolved with servers in `nameservers` are in the specified
  # subnets below, they are considered invalid and results from `fallback`
  # servers are used instead.
  #
  # IP address resolved with servers in `nameserver` is used when
  # `fallback-filter.geoip` is true and when GEOIP of the IP address is `CN`.
  #
  # If `fallback-filter.geoip` is false, results from `nameserver` nameservers
  # are always used if not match `fallback-filter.ipcidr`.
  #
  # This is a countermeasure against DNS pollution attacks.
  # fallback-filter:
  #   geoip: true
  #   geoip-code: CN
  #   ipcidr:
  #     - 240.0.0.0/4
  #   domain:
  #     - '+.google.com'
  #     - '+.facebook.com'
  #     - '+.youtube.com'
  
  # Lookup domains via specific nameservers
  # nameserver-policy:
  #   'www.baidu.com': '114.114.114.114'
  #   '+.internal.crop.com': '10.0.0.1'

proxies:
  # Shadowsocks
  # The supported ciphers (encryption methods):
  #   aes-128-gcm aes-192-gcm aes-256-gcm
  #   aes-128-cfb aes-192-cfb aes-256-cfb
  #   aes-128-ctr aes-192-ctr aes-256-ctr
  #   rc4-md5 chacha20-ietf xchacha20
  #   chacha20-ietf-poly1305 xchacha20-ietf-poly1305
  - name: "ss1"
    type: ss
    server: server
    port: 443
    cipher: chacha20-ietf-poly1305
    password: "password"
    # udp: true

  - name: "ss2"
    type: ss
    server: server
    port: 443
    cipher: chacha20-ietf-poly1305
    password: "password"
    plugin: obfs
    plugin-opts:
      mode: tls # or http
      # host: bing.com

  - name: "ss3"
    type: ss
    server: server
    port: 443
    cipher: chacha20-ietf-poly1305
    password: "password"
    plugin: v2ray-plugin
    plugin-opts:
      mode: websocket # no QUIC now
      # tls: true # wss
      # skip-cert-verify: true
      # host: bing.com
      # path: "/"
      # mux: true
      # headers:
      #   custom: value

  # vmess
  # cipher support auto/aes-128-gcm/chacha20-poly1305/none
  - name: "vmess"
    type: vmess
    server: server
    port: 443
    uuid: uuid
    alterId: 32
    cipher: auto
    # udp: true
    # tls: true
    # skip-cert-verify: true
    # servername: example.com # priority over wss host
    # network: ws
    # ws-opts:
    #   path: /path
    #   headers:
    #     Host: v2ray.com
    #   max-early-data: 2048
    #   early-data-header-name: Sec-WebSocket-Protocol

  - name: "vmess-h2"
    type: vmess
    server: server
    port: 443
    uuid: uuid
    alterId: 32
    cipher: auto
    network: h2
    tls: true
    h2-opts:
      host:
        - http.example.com
        - http-alt.example.com
      path: /
  
  - name: "vmess-http"
    type: vmess
    server: server
    port: 443
    uuid: uuid
    alterId: 32
    cipher: auto
    # udp: true
    # network: http
    # http-opts:
    #   # method: "GET"
    #   # path:
    #   #   - '/'
    #   #   - '/video'
    #   # headers:
    #   #   Connection:
    #   #     - keep-alive

  - name: vmess-grpc
    server: server
    port: 443
    type: vmess
    uuid: uuid
    alterId: 32
    cipher: auto
    network: grpc
    tls: true
    servername: example.com
    # skip-cert-verify: true
    grpc-opts:
      grpc-service-name: "example"

  # socks5
  - name: "socks"
    type: socks5
    server: server
    port: 443
    # username: username
    # password: password
    # tls: true
    # skip-cert-verify: true
    # udp: true

  # http
  - name: "http"
    type: http
    server: server
    port: 443
    # username: username
    # password: password
    # tls: true # https
    # skip-cert-verify: true
    # sni: custom.com

  # Snell
  # Beware that there's currently no UDP support yet
  - name: "snell"
    type: snell
    server: server
    port: 44046
    psk: yourpsk
    # version: 2
    # obfs-opts:
      # mode: http # or tls
      # host: bing.com

  # Trojan
  - name: "trojan"
    type: trojan
    server: server
    port: 443
    password: yourpsk
    # udp: true
    # sni: example.com # aka server name
    # alpn:
    #   - h2
    #   - http/1.1
    # skip-cert-verify: true

  - name: trojan-grpc
    server: server
    port: 443
    type: trojan
    password: "example"
    network: grpc
    sni: example.com
    # skip-cert-verify: true
    udp: true
    grpc-opts:
      grpc-service-name: "example"

  - name: trojan-ws
    server: server
    port: 443
    type: trojan
    password: "example"
    network: ws
    sni: example.com
    # skip-cert-verify: true
    udp: true
    # ws-opts:
      # path: /path
      # headers:
      #   Host: example.com

  # ShadowsocksR
  # The supported ciphers (encryption methods): all stream ciphers in ss
  # The supported obfses:
  #   plain http_simple http_post
  #   random_head tls1.2_ticket_auth tls1.2_ticket_fastauth
  # The supported supported protocols:
  #   origin auth_sha1_v4 auth_aes128_md5
  #   auth_aes128_sha1 auth_chain_a auth_chain_b  
  - name: "ssr"
    type: ssr
    server: server
    port: 443
    cipher: chacha20-ietf
    password: "password"
    obfs: tls1.2_ticket_auth
    protocol: auth_sha1_v4
    # obfs-param: domain.tld
    # protocol-param: "#"
    # udp: true

proxy-groups:
  # relay chains the proxies. proxies shall not contain a relay. No UDP support.
  # Traffic: clash <-> http <-> vmess <-> ss1 <-> ss2 <-> Internet
  - name: "relay"
    type: relay
    proxies:
      - http
      - vmess
      - ss1
      - ss2

  # url-test select which proxy will be used by benchmarking speed to a URL.
  - name: "auto"
    type: url-test
    proxies:
      - ss1
      - ss2
      - vmess1
    # tolerance: 150
    # lazy: true
    url: 'http://www.gstatic.com/generate_204'
    interval: 300

  # fallback selects an available policy by priority. The availability is tested by accessing an URL, just like an auto url-test group.
  - name: "fallback-auto"
    type: fallback
    proxies:
      - ss1
      - ss2
      - vmess1
    url: 'http://www.gstatic.com/generate_204'
    interval: 300

  # load-balance: The request of the same eTLD+1 will be dial to the same proxy.
  - name: "load-balance"
    type: load-balance
    proxies:
      - ss1
      - ss2
      - vmess1
    url: 'http://www.gstatic.com/generate_204'
    interval: 300
    # strategy: consistent-hashing # or round-robin

  # select is used for selecting proxy or proxy group
  # you can use RESTful API to switch proxy is recommended for use in GUI.
  - name: Proxy
    type: select
    # disable-udp: true
    proxies:
      - ss1
      - ss2
      - vmess1
      - auto
 
  # direct to another infacename or fwmark, also supported on proxy
  - name: en1
    type: select
    interface-name: en1
    routing-mark: 6667
    proxies:
      - DIRECT 

  - name: UseProvider
    type: select
    use:
      - provider1
    proxies:
      - Proxy
      - DIRECT

proxy-providers:
  provider1:
    type: http
    url: "url"
    interval: 3600
    path: ./provider1.yaml
    health-check:
      enable: true
      interval: 600
      # lazy: true
      url: http://www.gstatic.com/generate_204
  test:
    type: file
    path: /test.yaml
    health-check:
      enable: true
      interval: 36000
      url: http://www.gstatic.com/generate_204

tunnels:
  # one line config
  - tcp/udp,127.0.0.1:6553,114.114.114.114:53,proxy
  - tcp,127.0.0.1:6666,rds.mysql.com:3306,vpn
  # full yaml config
  - network: [tcp, udp]
    address: 127.0.0.1:7777
    target: target.com
    proxy: proxy

rules:
  - DOMAIN-SUFFIX,google.com,auto
  - DOMAIN-KEYWORD,google,auto
  - DOMAIN,google.com,auto
  - DOMAIN-SUFFIX,ad.com,REJECT
  - SRC-IP-CIDR,192.168.1.201/32,DIRECT
  # optional param "no-resolve" for IP rules (GEOIP, IP-CIDR, IP-CIDR6)
  - IP-CIDR,127.0.0.0/8,DIRECT
  - GEOIP,CN,DIRECT
  - DST-PORT,80,DIRECT
  - SRC-PORT,7777,DIRECT
  - RULE-SET,apple,REJECT # Premium only
  - MATCH,auto
\end{lstlisting}



\end{document}